\documentclass[acmsmall, screen, review]{acmart}
\settopmatter{printfolios=true,printccs=true,printacmref=true}

%%
%% Bibliography and Citation Style
%%
\bibliographystyle{ACM-Reference-Format}
\citestyle{acmauthoryear}

\usepackage{amsmath}
\usepackage[nameinlink,noabbrev,capitalise]{cleveref}
\usepackage{mathtools}
\usepackage{thmtools}
\usepackage{bussproofs}
\usepackage{listings}
\usepackage{csquotes}
\usepackage{tikz}
\usepackage{MnSymbol}
\usepackage{graphicx}
\usepackage[mathscr]{euscript}
\usepackage{cmll}

%%
%% Todo notes
%%
\setlength{\marginparwidth}{2cm} % This is needed, otherwise the package "todonotes" will generate a warning.
\usepackage[]{todonotes}
\newcommand{\david}[2][]{\todo[inline,color=red!40,author=David,#1]{#2}}

%%
%% Macros
%%
\newcommand{\jdgmtname}[1]{\llangle\mathtt{#1}\rrangle}
\newcommand{\specialcolon}{:}

% Cp.: https://tex.stackexchange.com/questions/273034/square-version-of-cdot-small-black-square
\makeatletter
\DeclareRobustCommand{\sqcdot}{\mathbin{\mathpalette\morphic@sqcdot\relax}}
\newcommand{\morphic@sqcdot}[2]{%
  \sbox\z@{$\m@th#1\centerdot$}%
  \ht\z@=.33333\ht\z@
  \vcenter{\box\z@}%
}
\makeatother

\begin{document}
%%
%% Title information
%%
\title{A Mechanized Specification of the Haskell 2010 Type System in the Lean Proof Assistant}

%%
%% Keywords
%%
\keywords{Haskell}

%%
%% CCS Classification
%%

\begin{CCSXML}
  <ccs2012>
     <concept>
         <concept_id>10003752.10003753.10003754.10003733</concept_id>
         <concept_desc>Theory of computation~Lambda calculus</concept_desc>
         <concept_significance>500</concept_significance>
         </concept>
     <concept>
         <concept_id>10011007.10011006.10011041</concept_id>
         <concept_desc>Software and its engineering~Compilers</concept_desc>
         <concept_significance>500</concept_significance>
         </concept>
     <concept>
         <concept_id>10011007.10011006.10011008.10011024.10011027</concept_id>
         <concept_desc>Software and its engineering~Control structures</concept_desc>
         <concept_significance>300</concept_significance>
         </concept>
   </ccs2012>
\end{CCSXML}

\ccsdesc[500]{Theory of computation~Lambda calculus}
\ccsdesc[500]{Software and its engineering~Compilers}
\ccsdesc[300]{Software and its engineering~Control structures}

%%
%% Author: David Binder
%%
\author{David Binder}
\orcid{0000-0003-1272-0972}
\affiliation{
  \department{School of Computing}
  \institution{University of Kent}
  \city{Canterbury}
  \country{United Kingdom}
}
\email{D.Binder@kent.ac.uk}



%%
%% Abstract
%%
\begin{abstract}
  The Haskell language report defines what the Haskell language \emph{is}; it does so, however, using natural language which can often be ambiguous.
  This ambiguity can lead to divergent behaviour among different compilers, so what we really require is a mathematically precise
  specification of the language.
  Over 20 years ago Karl Filip Faxen presented a comprehensive set of formal typing rules for the entire Haskell 98 surface language.
  Specifying the rules for a complete language, and not only some core subset of it, is necessarily verbose and error-prone.
  To ensure that we do not accidentally introduce bugs into the specification, we present a \emph{mechanized} version of Faxen's rules, adapted to the Haskell 2010 language standard, in the proof assistant Lean.
\end{abstract}

\maketitle


%%
%% Section: Introduction
%%
\section{Introduction}
\label{sec:introduction}
This describes the differences w.r.t \cite{Faxen2002staticsemantics}.

%%
%% Section: Kinding
%%
\section{Kinding}
\label{sec:kinding}
% Macros for kinding judgements

%%
%% Bibliography
%%
\bibliography{bibliography/bibliography.bib, bibliography/ownpublications.bib}


\end{document}
