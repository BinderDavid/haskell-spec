\documentclass[acmsmall, screen, review]{acmart}
\settopmatter{printfolios=true,printccs=true,printacmref=true}

%%
%% Bibliography and Citation Style
%%
\bibliographystyle{ACM-Reference-Format}
\citestyle{acmauthoryear}

\usepackage{amsmath}
\usepackage[nameinlink,noabbrev,capitalise]{cleveref}
\usepackage{mathtools}
\usepackage{thmtools}
\usepackage{bussproofs}
\usepackage{listings}
\usepackage{csquotes}
\usepackage{tikz}
\usepackage{MnSymbol}
\usepackage{graphicx}
\usepackage[mathscr]{euscript}
\usepackage{cmll}

%%
%% Todo notes
%%
\setlength{\marginparwidth}{2cm} % This is needed, otherwise the package "todonotes" will generate a warning.
\usepackage[]{todonotes}
\newcommand{\david}[2][]{\todo[inline,color=red!40,author=David,#1]{#2}}

%%
%% Macros
%%
\newcommand{\jdgmtname}[1]{\llangle\mathtt{#1}\rrangle}
%% Macros for kinding judgements

%% Lean Versions
\newcommand{\jktype}[3]{\jdgmtname{ktype} #1 \vdash #2 \specialcolon #3\ \sqcdot}
\newcommand{\jkindord}[2]{\jdgmtname{kindord} #1 \leq #2 \sqcdot}
\newcommand{\jkindsig}[2]{\jdgmtname{ksig} #1 \vdash #2\ \sqcdot}

%% Faxen Versions
\newcommand{\fktype}[3]{ #1 \vdash^{\text{ktype}} #2 : #3}
\newcommand{\fkindsig}[2]{#1 \vdash^{\text{ksig}} #2}
\newcommand{\fkindsigs}[2]{#1 \vdash^{\text{ksigs}} #2}
\newcommand{\fkindctx}[2]{#1 \vdash^{\text{kctx}} #2}


\begin{document}
%%
%% Title information
%%
\title{A Mechanized Specification of the Haskell 2010 Type System in the Lean Proof Assistant}

%%
%% Keywords
%%
\keywords{Haskell}

%%
%% CCS Classification
%%

\begin{CCSXML}
  <ccs2012>
     <concept>
         <concept_id>10003752.10003753.10003754.10003733</concept_id>
         <concept_desc>Theory of computation~Lambda calculus</concept_desc>
         <concept_significance>500</concept_significance>
         </concept>
     <concept>
         <concept_id>10011007.10011006.10011041</concept_id>
         <concept_desc>Software and its engineering~Compilers</concept_desc>
         <concept_significance>500</concept_significance>
         </concept>
     <concept>
         <concept_id>10011007.10011006.10011008.10011024.10011027</concept_id>
         <concept_desc>Software and its engineering~Control structures</concept_desc>
         <concept_significance>300</concept_significance>
         </concept>
   </ccs2012>
\end{CCSXML}

\ccsdesc[500]{Theory of computation~Lambda calculus}
\ccsdesc[500]{Software and its engineering~Compilers}
\ccsdesc[300]{Software and its engineering~Control structures}

%%
%% Author: David Binder
%%
\author{David Binder}
\orcid{0000-0003-1272-0972}
\affiliation{
  \department{School of Computing}
  \institution{University of Kent}
  \city{Canterbury}
  \country{United Kingdom}
}
\email{D.Binder@kent.ac.uk}



%%
%% Abstract
%%
\begin{abstract}
  The Haskell language report defines what the Haskell language \emph{is}; it does so, however, using natural language which can often be ambiguous.
  This ambiguity can lead to divergent behaviour among different compilers, so what we really require is a mathematically precise
  specification of the language.
  Over 20 years ago Karl Filip Faxen presented a comprehensive set of formal typing rules for the entire Haskell 98 surface language.
  Specifying the rules for a complete language, and not only some core subset of it, is necessarily verbose and error-prone.
  To ensure that we do not accidentally introduce bugs into the specification, we present a \emph{mechanized} version of Faxen's rules, adapted to the Haskell 2010 language standard, in the proof assistant Lean.
\end{abstract}

\maketitle


%%
%% Section: Introduction
%%
\section{Introduction}
\label{sec:introduction}
This article describes a Lean mechanization of the typing system described in \cite{Faxen2002staticsemantics}.
In particular, we discuss the differences between the version of the typing rules described in that paper, and the rules formalized in the mechanization.
These differences are mainly due to the different underlying technologies: pen and paper on the one hand and a modern interactive proof assistant on the other hand.
These differences go both ways:
Some things which are easy to specify by hand are difficult to implement in the proof assistant, and some things which are easy to specify in a proof assistant, using higher-order combinators, for example, are cumbersome to write down in a paper.

\subsection{Notations for Judgment Forms}

\citet{Faxen2002staticsemantics} uses the following notation for typing judgements.
\begin{equation*}
    \textit{environments} \vdash^{\textit{judgement name}} \textit{source phrase} \leadsto \textit{target phrase} : \textit{derived information}
\end{equation*}
A judgement of this form uses information taken from an environment and elaborates a source phrase of the surface language to a target phrase of the System-F based intermediate language.
Additional derived information can be derived types or kinds, as well as new environments when checking toplevel declarations or bindings, for example.

In the Lean mechanization we write judgements as follows:
\begin{equation*}
    \jdgmtname{judgement name} \textit{environments} \vdash \textit{source phrase} \leadsto \textit{target phrase} \specialcolon \textit{derived information}\ \sqcdot
\end{equation*}
This difference is due to the technical constraints of proof assistants, in particular the extensible parsers for user defined notations:
The starting delimiter $\jdgmtname{judgement name}$ and ending delimiter $\sqcdot$ of the judgement form make parsing much simpler and allow it to omit many unnecessary parentheses.
Instead of the colon we use a special colon $\specialcolon$ to clearly distinguish it from the symbol that the specification language Lean uses for type ascriptions.

%%
%% Section: Kinding
%%
\section{Kinding}
\label{sec:kinding}
%% Macros for kinding judgements

%% Lean Versions
\newcommand{\jktype}[3]{\jdgmtname{ktype} #1 \vdash #2 \specialcolon #3\ \sqcdot}
\newcommand{\jkindord}[2]{\jdgmtname{kindord} #1 \leq #2 \sqcdot}
\newcommand{\jkindsig}[2]{\jdgmtname{ksig} #1 \vdash #2\ \sqcdot}

%% Faxen Versions
\newcommand{\fktype}[3]{ #1 \vdash^{\text{ktype}} #2 : #3}
\newcommand{\fkindsig}[2]{#1 \vdash^{\text{ksig}} #2}
\newcommand{\fkindsigs}[2]{#1 \vdash^{\text{ksigs}} #2}
\newcommand{\fkindctx}[2]{#1 \vdash^{\text{kctx}} #2}


%%
%% Bibliography
%%
\bibliography{bibliography/bibliography.bib, bibliography/ownpublications.bib}


\end{document}
