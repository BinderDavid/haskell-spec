This article describes a Lean mechanization of the typing system described in \cite{Faxen2002staticsemantics}.
In particular, we discuss the differences between the version of the typing rules described in that paper, and the rules formalized in the mechanization.
These differences are mainly due to the different underlying technologies: pen and paper on the one hand and a modern interactive proof assistant on the other hand.
These differences go both ways:
Some things which are easy to specify by hand are difficult to implement in the proof assistant, and some things which are easy to specify in a proof assistant, using higher-order combinators, for example, are cumbersome to write down in a paper.

\subsection{Notations for Judgment Forms}

\citet{Faxen2002staticsemantics} uses the following notation for typing judgements.
\begin{equation*}
    \textit{environments} \vdash^{\textit{judgement name}} \textit{source phrase} \leadsto \textit{target phrase} : \textit{derived information}
\end{equation*}
A judgement of this form uses information taken from an environment and elaborates a source phrase of the surface language to a target phrase of the System-F based intermediate language.
Additional derived information can be derived types or kinds, as well as new environments when checking toplevel declarations or bindings, for example.

In the Lean mechanization we write judgements as follows:
\begin{equation*}
    \jdgmtname{judgement name} \textit{environments} \vdash \textit{source phrase} \leadsto \textit{target phrase} \specialcolon \textit{derived information}\ \sqcdot
\end{equation*}
This difference is due to the technical constraints of proof assistants, in particular the extensible parsers for user defined notations:
The starting delimiter $\jdgmtname{judgement name}$ and ending delimiter $\sqcdot$ of the judgement form make parsing much simpler and allow it to omit many unnecessary parentheses.
Instead of the colon we use a special colon $\specialcolon$ to clearly distinguish it from the symbol that the specification language Lean uses for type ascriptions.